\documentclass[10pt]{article}
\usepackage{graphicx, verbatim}
\usepackage{amsmath}
\usepackage{amssymb}
\usepackage{amscd}
\usepackage{lipsum}
\usepackage{blindtext}
\usepackage{todonotes}
\usepackage[tableposition=top]{caption}
\usepackage{ifthen}
\usepackage[utf8]{inputenc}
\usepackage{graphicx}
\usepackage{caption}
\setlength{\textwidth}{6.5in} 
\setlength{\textheight}{9in}
\setlength{\oddsidemargin}{0in} 
\setlength{\evensidemargin}{0in}
\setlength{\topmargin}{-1.5cm}
\setlength{\parindent}{0cm}
\usepackage{setspace}
\usepackage{float}
\usepackage{amssymb}
\usepackage[utf8]{inputenc}
\usepackage{fancyhdr}
\usepackage{tabularx}

\usepackage{hyperref}
\hypersetup{
  colorlinks   = true, %Colours links instead of ugly boxes
  urlcolor     = blue, %Colour for external hyperlinks
  linkcolor    = blue, %Colour of internal links
  citecolor   = red %Colour of citations
}
%\usepackage[backend=bibtex ,sorting=none]{biblatex}
\usepackage[backend=biber ,sorting=none]{biblatex}
\bibliography{references}

\begin{filecontents*}{references.bib}

\end{filecontents*}

%\usepackage[backend=biber]{biblatex}
%\addbibresource{references.bib}


%\fancyhf{}
\rfoot{Group 2 \thepage}
\singlespacing
\usepackage[affil-it]{authblk} 
\usepackage{etoolbox}
\usepackage{lmodern}

% \makeatletter
% \renewcommand{\maketitle}{\bgroup\setlength{\parindent}{16pt}
% \begin{flushleft}
%   \textbf{\@title}
% 
%   \@author
% \end{flushleft}\egroup
% }

%\renewcommand\Authfont{\fontsize{14}{18.4}\selectfont}
%\makeatother

% \pagestyle{fancy}
% \rfoot{Page \thepage}
 %\thispagestyle{empty} 


\begin{document}


\title{\LARGE Plastic Pollution in Oceans  \\ Group 2 Report - CMM507}

\author{ALEXANDER RITCHIE, \textit{\href{1911218@rgu.ac.uk}{1911218@rgu.ac.uk}}; GEORGIOS ORFANAKIS, \textit{\href{1903446@rgu.ac.uk}{1903446@rgu.ac.uk}};KAREN JEWELL, \textit{\href{1415410@rgu.ac.uk}{1415410@rgu.ac.uk}};ROSHI SHRESTHA, \textit{\href{1903445@rgu.ac.uk}{1903445@rgu.ac.uk}};STUART WATT, \textit{\href{1501869@rgu.ac.uk}{1501869@rgu.ac.uk}}}
%ALEXANDER RITCHIE (1911218) ;GEORGIOS ORFANAKIS (1903446); KAREN JEWELL (1415410); ROSHI SHRESTHA (1903445); STUART WATT (1501869); 

\maketitle
% \begin{flushleft} \today \end{flushleft} 
\noindent\rule{16cm}{0.4pt}
%\underline{\hspace{3cm}
\ \\
%\thispagestyle{empty}

\section*{Objective}


\begin{itemize}
\item To understand the composition of plastic pollutants in the ocean
\item To understand the sources of plastic pollutants
\item To understand how plastic pollution gets distributed across the oceans
\end{itemize}

\section{Problem Statement}\label{statement}

H1 = The \% of plastic pollution remains constant over time.

H0 = The \% of plastic pollution does not remain constant over time.


\subsection{Overview}\label{over}

Marine pollution is a major global issue which impacts on environment, economy and human health. Although marine pollution is caused by many different materials, plastics consist of 60-80\% of the marine litter (Derraik, 2002; Reisser, 2015, Barboza et al., 2019; Tekman et al., 2019). 

Synthetic organic polymer derived from polymerisation of monomers extracted from oil and gas make up the plastics (Derraik, 2002; Rios etal., 2007). Plastic has been the most used manmade materials since the 1900s. Since the 1940s mass production of plastic has been increasing rapidly worldwide (PlasticsEurope, 2010). The lightweight feature and its durability make it very suitable to make a range of products that we use in our everyday life (Barnes et al., 2009; Sivan 2011). These same features have been a major cause of pollution due to overuse and non-managed waste disposal system worldwide with plastic contributing to the 10\% of the waste generated worldwide (Barnes et al., 2009). Due to its buoyancy, plastic debris can be dispersed over long distances and they can persist for a long time (Goldberg, 1997). Although, plastic litter has been a major cause of marine pollution for a while, its seriousness has only been realised recently (Stefatos et al., 1999). 


\subsection{Motivation}\label{mot}
Jambeck et al., (2015) reported that in 2010 alone, between 4.8 million to 12.7 million metric tons of plastics entered the ocean. Plastics are now everywhere in the marine environment and urgent action is required to mitigate this problem and reduce the harmful impact (Rios et al., 2007; Rochman et al., 2015). 

Various papers covering the broad range of topics related to marine litter will be discussed. (this will be topics for literature review?? To be added later)

Impact on marine life

Plastics in ocean is of the many forms of human impact that threatens marine life. There is still very little information available on the impact of plastic pollution on the ocean's ecosystem. Due to the realisation on impact of human on climate and environment, there has been a lot of awareness activities to reduce the impact of pollution. Ban on single use plastic bags are being applied to many countries now (ref???) in order to protect the environment. 

Over 700 marine organisms are affected due to entanglement in plastic ropes and materials and ingestion of plastics in the ocean (Reff?). Over 340 species of marine animals were found to be entangled (Kuhn et al., ). In UK 2-9\% of animals were affected by entanglement (Werner et al., 2016). Reducing plastic waste is a major challenge worldwide. It is almost impossible to estimate the number of marine animals affected by marine pollution globally due to the vastness of the ocean. However, studies carried out on the gut contents of thousands of seabirds, found the significant increase in the ingestion of plastics during the 10-15 years interval (Robards et al., 1995). This result might correlate to the rapid increase of plastic production and plastic use globally.  In a study carried out over fourteen years, Moser and Lee (1992) found that more 50\% of the seabird species contained plastic particles in the gut which increased over time. This could be due the increase in plastic availability over time. 
Entanglement in plastic debris is another cause of marine life suffering. Discarded fishing gear and floating mastic masses in ocean are serious threat to marine animals. Some animals such as seals are attracted to the floating plastics where they get entangled and suffocated (Mattlin and Cawthron, 1986). Floating plastics over long distances can disperse alien species as well as some pathogens. Drifting plastic debris are also the source of alien species introduction and thus affecting the native marine biodiversity (McKinney, 1998). 

Impact on environment

Plastic debris floating in the oceans and the littering the coastal areas are not a pleasant sight. Masses of plastic accumulation and discarded objects made from plastics are found everywhere nowadays. (affecting the recreational activities and thus impacting on socio-economic status in some areas: to be discussed later).
Impact on human health
Over time plastic disintegrates into small microplastics which are easily consumed by fish and they enter the food chain. Plastics have been found in a third of fish caught in the UK which included the popular fishes such as cod, haddock and mackerel (reff??). Impact of plastic entering the human food chain and the effects of it are still to be studied.  Plastic toxicity and the occurrence of microplastics and nanoplastics in the water supply can also be a direct impact on human health in addition to the contamination in seafood (Rochman et al., 2015). 


Sources of marine plastic pollution

https://pubs.acs.org/doi/10.1021/acs.est.7b02368 plastic debris from river to sea

Around 80\% of the 8 million tonnes of plastics come from land-based sources, with the remainder coming from shipping and the fishing industry (refff>???). 

Microplastic in food. Microplastic are formed by fragmentation on the large plastic debris...
https://www.ncbi.nlm.nih.gov/pmc/articles/PMC6132564/



\subsection{Objectives }\label{obj}

The main objectives of this project can be outlined as follows: 




\section{Research}\label{research}

Things we found
citation example \cite{8489087}.

Sources of pollution: 10 river dataset, 50km2 coastline dataset, pollution density and body of water dataset....


\section {Methods}\label{methods}

This paper is conducted using secondary data collection methods only. The authors did not collect or create any new data using primary methods.


\subsection{Dataset Description}\label{dataset}

\begin{itemize}
%\Where the dataset came from; How it is constructed: multiple csv files by year; A description of what it is, what's in it and what it represents; Problems with the dataset: Missing data; data anomalies (lat/long values don't match named regions)

\item The data was taken from marine debris tracker (marinedebris.engr.uga.edu/newmap/) between 2010 till February 19th 2020. The time of 2010 was chosen as there was no data before that time.
\item The dataset was composed by combining the multiple csv files gathered from the marine debris tracker into a single set after this was done the date data type was renamed "Time". 
\item The dataset created from the combined csv files contain more than 360000 rows of data and consists of the folowing variables.
  \begin{itemize}
  \item ListID is the ID code for the list
  \item ListName is the name of the list
  \item ItemID is the ID code given to the item of debris
  \item ItemName is the name we give to item of debris
  \item LogID is the ID code given to the location of the debris
  \item Latitude, Longitude and Altitude are the coordinates of the location where the observation was made
  \item Quantity is the number of pieces of debris in the observation.
  \item Error radius is the radius around the observation site within the error for reasonable doubt.
  \item Location is the area the observation of debris was made in.
  \item Description is the description of the area the debris was found in.
  \item MaterialID is the ID code of the material that the debris was composed of. 
  \item Material Description is the description given to the material that composes the debris.
  \item Time is the time that the observation was made. 
\item There were a number of problems with the dataset namely;
  \begin{itemize}
  \item There were a number of cases of missing data in the dataset. 
  \item data anomalies (lat/long values don't match named regions)
  \item 
  \end{itemize}
\end{itemize}


\subsection{Dataset Pre-processing}
%\Because of the features and concerns identified in the section above, we chose to transform the dataset in the following ways:reclassified some labels because variation was too high (there were too many labels); removed missing values; removed certain subsets; but kept certain subsets

Everything below is from Stuart's RNW file

Logged marine debris is available for download \textit{\href{http://marinedebris.engr.uga.edu/newmap/}{here}}. I'm importing data from 2010 till Feb 19th 2020. There doesn't seem to be data before 2010. The data is reported marine debris.
DataImport


























































